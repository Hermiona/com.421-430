\documentclass[12pt,a4paper,oneside]{article}

\usepackage{hyperref}
\hypersetup{
    pdftitle={COM 421, Software Engineering},%
    pdfauthor={Toksaitov Dmitrii Alexandrovich},%
    pdfsubject={Syllabus},%
    pdfkeywords={COM;}{421;}{syllabus;}{software;}{engineering},%
    colorlinks,%
    linkcolor=black,%
    citecolor=black,%
    filecolor=black,%
    urlcolor=black
}

\begin{document}

    \title{COM 421, Software Engineering}
    \author{
        American University of Central Asia\\
        Software Engineering Department
    }
    \date{}
    \maketitle

    \section{Course Information}

        \begin{description}
            \item[Course ID]\hfill\\
                COM 421, 3705
            \item[Place]\hfill\\
                AUCA new campus, room G30
            \item[Time]\hfill\\
                Monday 12:45\\
                Wednesday 12:45
        \end{description}

    \section{Prerequisites}

        COM 112, Programming \uppercase\expandafter{\romannumeral 2\relax}. Object Oriented Design and GUI Programming
        COM 324, Algorithm Analysis

        \section{Contact Information}

            \begin{description}
                \item[Instructor]\hfill\\
                    Toksaitov Dmitrii Alexandrovich\\
                    \href{mailto:toksaitov_d@auca.kg}{toksaitov\_d@auca.kg}
                \item[Office]\hfill\\
                    AUCA new campus, room 3xx (TBD)
                \item[Office Hours]\hfill\\
                    \begin{itemize}
                        \item Monday 10:40--12:45
                        \item Wednesday 14:00--15:00
                    \end{itemize}
        \end{description}

    \section{Course Overview}

        This course introduces students to software engineering, teaching each of the individual steps of the software life cycle: requirements, design, coding, testing and delivery. The course also covers estimating man months to complete a project and writing project proposals. Along with the theory, students will go through all the stages of software development on their own project. This is a two-semester course designed for Software Engineering Majors and Minors.

    \section{Practice Tasks}

        Students are required to finish 10 practice tasks during the course. This tasks are based on topics discussed during lectures. Each task should be finished during the class to receive a grade.

    \section{Course Project}

        The main project for the course is a team development of a software product for an imaginary customer. Small teams of 4 people will compete to deliver solutions to a customer's problem. Teams will go through all steps of software production such as requirements specification, software design, construction, testing, deployment, and maintenance.

    \section{Reading}

        Software Engineering (9'th Edition) by Ian Sommerville (AUCA Library Call Number: QA76.758.S657 2011, ISBN: 9780137035151)

    \section{Grading}

        \begin{itemize}
            \item Practice tasks (40\%)
            \item Course project (several parts)
            \begin{itemize}
                \item Major part of the final grade (60\%)
            \end{itemize}
        \end{itemize}

        \newpage

        \begin{itemize} \itemsep-10pt \parskip0pt \parsep0pt
            \item[--] 90\%--100\%: A\\
            \item[--] 80\%--89\%: A-\\
            \item[--] 70\%--79\%: B+\\
            \item[--] 65\%--69\%: B\\
            \item[--] 60\%--64\%: B-\\
            \item[--] 56\%--59\%: C+\\
            \item[--] 53\%--55\%: C\\
            \item[--] 50\%--52\%: C-\\
            \item[--] 46\%--49\%: D+\\
            \item[--] 43\%--45\%: D\\
            \item[--] 40\%--42\%: D-\\
            \item[--] Less than 39\%: F
        \end{itemize}

    \section{Rules}

        Students are required to follow the rules of conduct of the Software Engineering Department and American University of Central Asia.

\end{document}
