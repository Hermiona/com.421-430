\documentclass[12pt,a4paper,oneside]{article}

\usepackage[margin=3cm]{geometry}

\usepackage{hyperref}
\hypersetup{
    pdftitle={COM 421, 430, Software Engineering I, II},%
    pdfauthor={Toksaitov Dmitrii Alexandrovich},%
    pdfsubject={Syllabus},%
    pdfkeywords={COM;}{421;}{430;}{syllabus;}{software;}{engineering},%
    colorlinks,%
    linkcolor=black,%
    citecolor=black,%
    filecolor=black,%
    urlcolor=black
}

\newcommand{\R}[1]{\uppercase\expandafter{\romannumeral #1\relax}}

\begin{document}

    \title{COM 421, 430, Software Engineering \R{1}, \R{2}}
    \author{
        American University of Central Asia\\
        Software Engineering Department
    }
    \date{}
    \maketitle

    \section{Course Information}

        \begin{description}
            \item[Course IDs]\hfill\\
                COM 421, 3705\\
                COM 430, 3881
            \item[Course Repository]\hfill\\
                \url{https://github.com/auca/com.421-430}
            \item[Place]\hfill\\
                AUCA, laboratory G30
            \item[Time]\hfill\\
                Monday 10:50\\
                Wednesday 9:25
        \end{description}

    \section{Prerequisites}

        COM 112, Programming \R{2}. Object Oriented Design and GUI Programming\\
        COM 324, Algorithm Analysis

        \section{Contact Information}

            \begin{description}
                \item[Instructor]\hfill\\
                    Toksaitov Dmitrii Alexandrovich\\
                    \href{mailto:toksaitov_d@auca.kg}{toksaitov\_d@auca.kg}
                \item[Office]\hfill\\
                    AUCA new campus, room 315
                \item[Office Hours]\hfill\\
                    Monday 12:45--14:45\\
                    Wednesday 12:45--14:45\\
                    Friday 14:00--16:00
            \end{description}

    \section{Course Overview}

        The course introduces students to software engineering, teaching each of
        the individual steps of the software life cycle: requirements, design,
        coding, testing and software delivery. The course covers estimating man
        months to complete a project and writing project proposals. Along with
        the theory, students will go through all the stages of software
        development on their own projects. This is a two-semester course
        designed for Software Engineering majors and minors.

    \section{Topics Covered}

        \begin{itemize}
            \item Software engineering concepts
            \item Software development methodologies
            \item Agile software development
            \item Requirements engineering
            \item System design and modeling
            \item System implementation
            \item Software testing
            \item Software evolution
            \item Project management
            \item Dependability and security
        \end{itemize}

    \section{Practice Tasks}

        Students are required to finish 6 practice tasks. The tasks are based on
        topics discussed during lectures. Each task should be finished during
        the class to receive a grade.

    \section{Course Projects}

        The course contains two projects for each semester. Each project
        requires to develop a software product for an imaginary customer. Small
        teams of 4 students will compete to deliver solutions to a specified
        problem. Teams will go through all steps of software production such as
        requirements specification, software design, construction, testing,
        deployment, and maintenance. Students will get a chance to practice
        agile methodologies such as Scrum and Kanban, work with Version Control
        Systems (VCS) such as Git or Mercurial, work with project management
        systems, learn to test their systems, try out Test-Driven Development,
        practice Continuous Integration (CI) and Continuous Delivery (CD).

    \section{Final Quizzes}

        At the end of each semester students will get a quiz on topics discussed
        during the course.

    \section{Reading}

        Software Engineering (9'th Edition) by Ian Sommerville (AUCA Library
        Call Number: QA76.758.S657 2011, ISBN: 978-0137035151)

            \subsection{Supplemental Reading}

                \begin{enumerate}
                    \item The Mythical Man-Month: Essays on Software
                    Engineering, Second Edition by Frederick P. Brooks Jr.
                    (ISBN: 858-0001065793)
                    \item Extreme Programming Explained: Embrace Change, 2nd
                    Edition by Kent Beck, Cynthia Andres (ISBN: 978-0321278654)
                    \item Essential Scrum: A Practical Guide to the Most Popular
                    Agile Process by Kenneth S. Rubin (ISBN: 007-6092046028)
                    \item Test Driven Development: By Example by Kent Beck
                    (ISBN: 978-0321146533)
                    \item Code Complete: A Practical Handbook of Software
                    Construction by Steve McConnell (AUCA Library Call Number:
                    QA76.76.D47M39 2004, ISBN: 079-0145196705)
                    \item Design Patterns: Elements of Reusable Object-Oriented
                    Software by Erich Gamma, Richard Helm, Ralph Johnson, John
                    Vlissides (AUCA Library Call Number: QA 76.64 D47 1995,
                    ISBN: 978-0201633610)
                    \item Refactoring: Improving the Design of Existing Code by
                    Martin Fowler, Kent Beck, John Brant, William Opdyke, Don
                    Roberts (AUCA Library Call Number: QA76.76.R42 F695 1999,
                    ISBN: 978-0201485677)
                \end{enumerate}

    \section{Grading}

        \begin{itemize}
            \item Practice tasks (20\%)
            \item Course project (several parts)
            \begin{itemize}
                \item Major part of the final grade (60\%)
            \end{itemize}
            \item Final quiz (20\%)
        \end{itemize}

        \begin{itemize} \itemsep-10pt \parskip0pt \parsep0pt
            \item[--] 90\%--100\%: A\\
            \item[--] 80\%--89\%: A-\\
            \item[--] 70\%--79\%: B+\\
            \item[--] 65\%--69\%: B\\
            \item[--] 60\%--64\%: B-\\
            \item[--] 56\%--59\%: C+\\
            \item[--] 53\%--55\%: C\\
            \item[--] 50\%--52\%: C-\\
            \item[--] 46\%--49\%: D+\\
            \item[--] 43\%--45\%: D\\
            \item[--] 40\%--42\%: D-\\
            \item[--] Less than 39\%: F
        \end{itemize}

    \section{Rules}

        Students are required to follow the rules of conduct of the Software
        Engineering Department and American University of Central Asia.

\end{document}

